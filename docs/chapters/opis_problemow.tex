\chapter{Napotkane problemy}

% opis napotkanych problemów, sposób ich rozwiązania...

Podczas realizacji projektu napotkaliśmy następujące problemy do rozwiązania:
    \begin{itemize}
        \item przejście do nowej linii
        \item translacja tabeli
    \end{itemize}

\section{Przejście do nowej linii}

W języku LateX przejście do nowej linii jest możliwe dzięki następujący sposób: wykorzystując znacznik newline, podwójnemu backslashowi, 
oraz wykonaniem podwójnego prześcia do nowej linii za pomocą klawisza Enter. W rezultacie chcieliśmy otrzymać znacznik "br" w HTMLu. 
Nie napotkaliśmy większych problemów z translacją znacznika "newline". Podwójny backslash również działał poprawnie dopóki nie 
zaimplementowaliśmy translacji tabeli, w której ten sam znacznik oznacza jej koniec. Ze względu na ten problem musieliśmy z niego 
zrezygnować. Problem nastąpił również z przejściem do nowej linii przy pomocy dwukrotnego wciśniecia klawisza Enter. W tym przypadku 
każde przejście do nowej linii było traktowane jako przejście pojedyncze, a podwójne było pomijane. W rezultacie oznaczało to, że nie 
mogliśmy otrzymać znacznika "br" w HTMLu. Po usunięciu definicji tokena "newline", znacznik "br" był produkowany, lecz program 
generował ostrzeżenie (brak znajomości tokena) w przypadku wystąpienia pojedynczego przejścia. Plik w formacie HTML był jednak 
poprawnie generowany. Ze względu na ostrzeżenie, pomysł został porzucony.

\section{Translacja tabeli}

Podstawowym problemem był opisany powyżej znacznik przejścia do nowej linii, który w przypadku tabeli oznacza jej koniec. Ponadto 
tabela jest na tyle złożoną sktrukturą, że nie jesteśmy w stanie zapewnić obsługi każdego z jej przypadków. Uprościliśmy więc translację 
do tabeli w pełni obramowanej lub nieobramowanej.