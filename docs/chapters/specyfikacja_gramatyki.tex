\chapter{Specyfikacja gramatyki}

\section{Tekst}
Nie wyszczególniono specjalnych wymagań, które musi spełniać zwykły tekst w języku \LaTeX \space, 
aby został przetłumaczony na język HTML.

\section{Formatowanie tekstu}

Translator obsługuje następujące znaczniki formatu LaTeX odpowiadające na formatowanie tekstu:
\begin{itemize}
    \item \textit{$bold$} - pogrubienie
    \item \textit{$italic$} - kursywa
    \item \textit{$underline$} - podkreślenie
    \item \textit{$centerline$} - wyśrodkowanie
    \item \textit{$tabulator$} - wcięcie
    \item \textit{$newline$} - nowa linia
    \item \textit{$title$} - tytuł
\end{itemize}

\section{Tabela}
W celu przetłumaczenia tabeli z języka \LaTeX \space na język HTML tabela powinna spełniać powniższe założenia: 
\begin{itemize}
    \item musi być zadeklarowana poleceniem \textit{$\backslash$begin\{tabular\}} oraz zakończona \textit{$\backslash$end\{tabular\}}
    \item po deklaracji \textit{$\backslash$begin\{tabular\}} musi zostać od razu zadeklarowane obramowanie wg standardu \LaTeX \space,
    np. \textit{$\backslash$begin\{tabular\}\{c c c\}}
    \item tabele z obramowaniem muszą zostać zadeklarowane wg \textit{$\backslash$begin\{tabular\}\{$|$ c $|$ c $|$ c $|$\}} - ilość kolumn dowolna
    \item tabele bez obramowania muszą zostać zadeklarowane wg \textit{$\backslash$begin\{tabular\}\{c c c\}}  - ilość kolumn dowolna
\end{itemize}

\section{Wyliczenie}

Program obsługuje wyliczenie uporządkowane i nieuporządkowane:
\begin{itemize}
    \item \textit{$backslash$begin\{enumerate\}} - uporządkowane
    \item \textit{$backslash$begin\{itemize\}} - nieuporządkowane
\end{itemize}

\section{Grafika}
Grafikę w tekście możemy zadeklarować wyłącznie używając polecenia \textit{$\backslash$includegraphics} podając ściezkę do zdjęcia np: 
\textit{$\backslash$includegraphics\{corgi.jpg\}}.

\section{Hiperłącze}

Hiperłącze w deklarujemy wyłącznie poprzez użycie \textit{$\backslash$url} np. \textit{$\backslash$url\{https://github.com/bartoszkordek/AGH-Jezyki-formalne-i-kompilatory\}}

\section{Sekcja, podsekcja, podpodsekcja}
Sekcje, podsekcje oraz podpodsekcje deklarujemy w ten sam sposób jak 
w języku \LaTeX \space np.  \textit{$\backslash$subsection\{Przykladowa podsekcja\}}.


% TODO 
% specyfikację gramatyki języka w notacji wybranego przez siebie narzędzia 
% wzorowane na Rozdziale 1 oraz Dodatku „Przewodnik języka C” (Sekcja „Gramatyka”) 
% książki „Język C”, B. Kernighan, D. Ritchie.