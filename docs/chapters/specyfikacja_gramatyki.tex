\chapter{Specyfikacja gramatyki}

\section{Tekst}

\section{Formatowanie tekstu}

Translator obsługuje następujące znaczniki formatu LaTeX odpowiadające na formatowanie tekstu:
\begin{itemize}
    \item \textit{$bold$} - pogrubienie
    \item \textit{$italic$} - kursywa
    \item \textit{$underline$} - podkreślenie
    \item \textit{$centerline$} - wyśrodkowanie
    \item \textit{$tabulator$} - wcięcie
    \item \textit{$newline$} - nowa linia
    \item \textit{$title$} - tytuł
\end{itemize}

\section{Tabela}

\section{Wyliczenie}

Program obsługuje wyliczenie uporządkowane i nieuporządkowane:
\begin{itemize}
    \item \textit{$backslash$begin\{enumerate\}} - uporządkowane
    \item \textit{$backslash$begin\{itemize\}} - nieuporządkowane
\end{itemize}

\section{Grafika}

\section{Hiperłącze}

Hiperłącze w deklarujemy wyłącznie poprzez użycie \textit{$\backslash$url} np. \textit{$\backslash$url\{https://github.com/bartoszkordek/AGH-Jezyki-formalne-i-kompilatory\}}

\section{Sekcja, podsekcja, podpodsekcja}


% TODO 
% specyfikację gramatyki języka w notacji wybranego przez siebie narzędzia 
% wzorowane na Rozdziale 1 oraz Dodatku „Przewodnik języka C” (Sekcja „Gramatyka”) 
% książki „Język C”, B. Kernighan, D. Ritchie.