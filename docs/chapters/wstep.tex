\chapter{Wstęp}

Celem projektu było zaimplementowanie translatora języka \LaTeX \space do HTML. W tym celu skorzystaliśmy z gotowej
biblioteki PLY \cite{ply} do skanowania i parsowania napisanej w języku Python. Kod odpowiadający za wykonanie translacji 
podzieliliśmy na dwie części: lekser (skanowanie) i parser (parsowanie). 

Do wykonania projektu skorzystaliśmy również z oficjalnej dokumentacji języka \LaTeX \space \cite{overleaf} 
oraz języka HTML \cite{html}.

\section{Pliki tworzące projekt}
Translator został zaimplementowany w języku Python 3.
Projekt składa się z następujących plików:
\begin{itemize}
    \item main.py - plik wejściowy programu, odpowiada za uruchomienie interpretera, wyświetlanie pomocniczych informacji w trybie debugowania
        oraz za uruchomienie odpowiedniego trybu działania programu
    \item theLexer.py - lexer, który ma za zadanie zwrócić tokeny po przeanalizowaniu kodu dokumentu latex
    \item theParser.py - parser, który ma za zadanie stworzyć abstrakcyjne drzewo syntaktyczne oraz zwrócic kod html
    \item example.tex - przykładowy plik testowy, który zostanie przetłumaczony do języka html
\end{itemize}

\section{Uruchamianie}
W celu uruchomienia tranlatora trzeba mieć zainstalowany interpreter języka Python 3 oraz 
w folderze projektu wpisać w terminalu komendę pokazaną na listingu \ref{uruchomienie}:
\begin{lstlisting}[language={Python}, caption={Uruchomienie}, label={uruchomienie}]
    python3 main.py -i example -o example
\end{lstlisting}

Parametr \textit{i} przyjmuje nazwę pliku do translacji, natomiast paramentr \textit{o} przyjmuje nazwę pliki html, który
zostanie stworzony podczas działania programu.